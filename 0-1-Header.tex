% documento per A4
\documentclass[a4paper,12pt,twoside, openright]{book}
%\usepackage{times} % usa i font PostScript
%\usepackage[italian]{babel} % sillabazione italiana
%Usare \hyphenation [] per specificare la sillabazione di una parola specifica
\usepackage[utf8]{inputenc}
\usepackage[a4paper,top=2.7cm,bottom=2.9cm,outer=2.7cm,inner=3.7cm, footskip=1.1cm]{geometry}
\usepackage[dvipsnames]{xcolor}

%\usepackage{cite}
\usepackage{indentfirst} % indenta la prima riga dei capitoli
%\setlength{\parindent}{0pt}

\usepackage{hyperref} % collegamenti ipertestuali e segnalibri
\hypersetup{
	colorlinks=true,
	bookmarksopen =true,
	bookmarksnumbered=true,
	linkcolor=black,
	anchorcolor=black,
	citecolor=black,
	urlcolor=blue,
	pdftitle={An \textit{in silico} investigation of the configurational entropy upon MBP-Maltose complexation},
	pdfauthor={Daniele Di Bari},
}

\urlstyle{same}

\usepackage{bookmark} % gestire i bookmarks

%Sources
%\usepackage{apacite}
%\bibliographystyle{apacite}
%\usepackage[nottoc]{tocbibind}

%immagini
\usepackage{tikz} 
\usepackage{graphicx,url}
\usepackage{sidecap}
%\usepackage{wrapfig}
\usepackage{float}
\usepackage[bf, small]{caption}
\usepackage{subcaption}
%\usepackage{subfigure} 
\usepackage{enumerate}

\graphicspath{	
	{./images/}
	{./images/chap01/}
	{./images/chap02/}
	{./images/chap03/}
}

%tabelle
\usepackage{tabularx}
\usepackage{multirow}
\usepackage{booktabs} % aggiunge: \toprule , \midrule e \bottomrule ; stile tabella
\usepackage{tabto} %aggiunge comando \tab
% per i grafici e simboli matematici
\usepackage{amsmath} 
\usepackage{amsfonts}
\usepackage{mathrsfs}
\usepackage{amssymb}
%\usepackage{eufrak}
\usepackage{mathptmx}
\usepackage{relsize}

\usepackage{fancyhdr} % impostare layout pagina - intestazione e pié di pagina
\pagestyle{fancy}

\usepackage{fontenc} %aggiunge font

%\fancyhead{} % clear all header fields
%\fancyfoot{} % clear all footer fields

%\fancyfoot[LE,RO]{\thepage} %imposta pagine non centrate | L-sx C-centro R-dx | E-pag pari O-pag dispari
%\renewcommand{\headrulewidth}{0pt} % spessore linea intestazione 
%\renewcommand{\footrulewidth}{0pt} % spessore linea pié pagina 

%\renewcommand{\thefootnote}{\fnsymbol{footnote}} % imposta come indici dei pié di pagina dei simboli 

\usepackage{epigraph} % epigrafe all'inizio dei capitoli
%\epigraphfontsize{\small}
\setlength\epigraphwidth{7.5cm}
\setlength\epigraphrule{0pt}

% Modifica interlinea:
\usepackage{setspace}
\singlespace % interlinea singola
%\onehalfspace % interlinea 1.5
%\doublespace % interlinea doppia

%\usepackage{lastpage}
\pagestyle{plain} % layout semplice - page into the foot centered

\setlength{\marginparwidth}{0cm} % imposta larghezza margini per le note a lato


\interfootnotelinepenalty=10000 %  più è grande il numero più è importante mantenere la nota a pié pagina di una singola pagina

\setcounter{tocdepth}{3}
\setcounter{secnumdepth}{3}

% colors
\definecolor{dred}{RGB}{200,0,0}

\newcommand{\changes}[1]{\textcolor{black}{#1}}