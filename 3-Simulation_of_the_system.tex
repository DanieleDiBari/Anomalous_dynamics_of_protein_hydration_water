\chapter{Results and methods}
In the present chapter, the original methodologies and results of the experimental process are displayed; they were collected and developed during the thesis research with the aim of devising MD simulations that are of relevant scientific interest. 
%Nel presente capitolo sono illustrate le metodologie ed i risultati del lavoro sperimentale svolto in maniera originale nel corso della tesi per sviluppare delle particolari simulazioni di dinamica molecolare di alcuni sistemi di interesse scientifico.

The systems that were mainly analyzed \textit{in silico} were powder solutions of proteins and water (i.e. solutions that are barely hydrated). The study of systems with a low hydration, is not only scientifically relevant, but it turns out to be very interesting from a perspective of the possible applications, for example for fields of applications that concern the study and projection of pharmaceuticals where the main active substance are proteins.\footnote{With the significant advancements in biologics and biopharmaceutical over the years, peptides and proteins have emerged with a host of new applications in the diagnostic as well as the therapeutic sector. As for the current calculations, the market for peptide and protein drugs is estimated around 10\% of the entire pharmaceutical market and will make up an even larger proportion of the market in the future. Since early 1980s, a total 239 therapeutic proteins and peptides are approved for clinical use by US-FDA (United States -- Food and Drug Administration [\url{10.1371/journal.pone.0181748}]}
%In particolare, i sistemi prevalentemente analizzati in silico sono soluzioni in polvere (ovvero poco idratate) di proteine e acqua. Lo studio di questo tipo di sistemi con una bassa percentuale di idratazione, oltre che da un punto di vista puramente scientifico, risulta molto ineteressante anche da un punto di vista applicativo per quanto riguarda, per esempio, campi di ricerca che concernono lo studio e la progettazione di farmaci in cui il principio attivo è costituito da proteine
 
For this reason, systems of proteins in powder are widely studied experimentally, principally through neutron spectroscopy. 
%Per questo, sistemi di proteine in polveri sono molto studiati sperimentalmente e, in particolare, questo viene fatto di solito attraverso la spettroscopia neutronica.
This technique, which is sensitive to the dynamics and the structure of condensed matter on the atomic scale, gives precise information about the mean atomic fluctuations and the dynamics in proteins, enhancing an average view of all atomic contributions.
However, due to the high complexity of these systems, usually their internal dynamics is too complicated for a quantitative interpretation at an atomistic detail starting from the average information obtained from neutron scattering.

In this context, computer simulations, and in particular Molecular Dynamics (MD) simulations, enable to gain a wider insight into the dynamics of proteins that are not provided from neutron scattering and from any real experiments, in general. 
Simulated trajectories can be analyzed in detail and information not accessible from experiments can be extracted from computer simulations.\footnote{For some interesting applications, on to the interpretation of neutron scattering, see the Kneller's lecture: ``\textit{Quasielastic Neutron Scattering}'' \cite{ref:QNS_Keller}.}

The objective of this thesis is precisely that of performing apo and holo simulations of a specific protein in powder environments that will enhance to collect valuable information about the interactions that contribute to the formation of the protein-ligand complex. In particular this protein and the ligand are:
%L'obbiettivo di questa tesi è infatti proprio quello di realizzare delle simulazioni delle forme apo e holo di una particolare proteina in ambienti polvesi che potranno permettere di ricavare preziose informazioni sull'interazioni che contribuiscono alla formazione del protein-ligand complex. Nello specifico, the protein and the ligand simulated are:
\begin{center}
\begin{minipage}{0.65\textwidth}
\begin{itemize}
\item[\textbf{Protein:}] Maltose-Binding Protein (MBP), also known as Maltosedextrin-Binding Protein
\vspace{-0.2cm}
\item[\textbf{Ligand:}] Maltose (Malt)
\end{itemize}
\end{minipage}
\end{center}
The MBP is a well studied model protein that plays an important role in the metabolism of Escherichia coli, e.g., in the energy-dependent translocation of maltose and maltodextrins through the cytoplasmic membrane \cite{paciaroni2008fingerprints}.\\
\\
INSERIRE IMMAGINE: proteina apo e proteina holo

%sebbene esistano diversi software per realizzare delle simulazioni di MD per studiare biomolecole, nel caso di sistemi scarsamente idratati   delle proteine proteine simulazioni di proteine in ambienti scarsamente idratati non sono abbasta

%Pertanto, nella prima sezione è descritta la realizzazione delle varie simulazioni realizzate nel corso della tesi. Nella seconda e nella terza sezione, invece, sono riportate alcune misure ottenute dell'analisi delle traiettorie simulate per studiare ...
In the first section, the implementation of the several simulations performed during the thesis research are described. 
In the second and third section the measurements obtained from the simulated trajectory analysis are reported...

\section{Simulations}
\subsection{Problems with powders}
In most part of the cases, the studied systems in biology are abundantly hydrated. For this reason, programs like NAMD [23], that allow to create accurate and efficient simulations of biological systems rapidly enough, are developed considering principally very diluted systems. In the case of the simulations of proteins in powders, the performing of simulations is notably much more difficult. In particular, due to the scarceness of water molecules, the building of the initial state can become quite cumbersome and difficult.
%Nella maggior parte dei casi, i sistemi studiati in ambito biologico sono abbondantemente idratati. Per questo motivo programmi come NAMD \cite{ref:NAMD}, che permettono di creare abbastanza rapidamente delle simulazioni di sistemi biologici il più possibile accurate ed efficienti, sono sviluppati considerando principalemente sistemi molto diluti. Nel caso delle simulazioni di proteine in polveri, tuttavia, la realizzazione delle simulazioni si complica notevolemente. In particolare, a causa della scarsità di molecole d'acqua, la costruzione dello stato iniziale può risultare abbastanza difficoltosa.

When studying the dynamics of proteins in very diluted systems, frequently it is sufficient to simulate a single protein in a water box that is big enough to avoid the auto-interaction of the protein with itself, caused by the periodic boundary conditions. In these systems, it is easy to construct the box with a setting where the water molecules that are randomly arranged inside the box, surround completely the protein and fulfill the box entirely in a quite uniform distribution.\\
%Per studiare la dinamica delle proteine in sistemi molto diluiti, di solito è sufficiente simulare una sola proteina in una box d'acqua abbastanza grande da evitare l'autointerazione della proteina con se stessa a causa delle condizioni di periodicità al contorno. In questi sistemi, è facile costruire queste box in modo tale che le molecole d'acqua, disposte casualemente all'interno della box, circondino totalmente la proteina e riempiono la completamente la box in modo abbastanza uniforme. \\
\\
INSERIRE IMMAGINE: esempio di sistema diluito\\
\\
On the other hand, in the case of powder solutions, due to the reduced number of water molecules, it is not possible to create such a box. Indeed, there are two possible scenarios:
Nel caso di soluzioni in porlvere, invece, a causa del ridotto numero di molecole d'acqua, non è possibile creare una box del genere. Si ha infatti che:
\begin{itemize}
\item[$\triangleright$] If the box is big enough to avoid that the protein interacts with itself due to the boundary conditions, the number of molecules within the box is not sufficient enough to fill up the box uniformly – empty regions, where no atoms are present, are created and the dimensions of these regions may be comparable to the protein itself. In this case, the initial state, more than just being very unstable, is far from representing a microstate of the real physical system.
%Se la box è abbastanza grande da evitare che la proteina interagisca con se stessa a causa delle boundary conditions, il numero di molecole al suo interno non è sufficiente a riempire la uniformemente la box -- si creano delle regioni vuote, dove non sono presenti atomi, delle dimensioni confrontabili con quelle della proteina. In questo caso lo stato iniziale, oltre ad essere molto insabile, è molto lontano dal rappresentare lo stato del sistema fisico reale.
\item[$\triangleright$] If a box with the dimensions of those of the protein is formed, the auto-interactions cannot be neglected, because they render the simulated system very different from the real one. Moreover, due to the irregular form of the protein and the scarcity of water molecules, also in this case the reduced dimensions of the box do not guarantee the absence of empty regions inside the box.
%Se si costruisce una box delle dimensioni pari circa a quelle della proteina, le auto interazioni sono tutt'altro che trascurabili rendendo il sistema simulato ampiamente diverso da quello reale che si intende studiare. Inoltre, a causa della forma inregolare della proteina e della scarsità di molecole d'acqua, anche in questo caso le ridotte dimensioni della box non garantirebbere l'assenza di vuoti all'interno della box.
\end{itemize} 

These type of complications imply that even if real experiments are performed in powder samples, very often, hydrated systems are however simulated for simplification. For example, in 2004, Balog et al. made an experiment of neutron scattering and showed that a particular protein in the holo form vibrates more intensively at lower frequencies (between 50 and 500 GHz) compared to the same protein in the apo form.
In 2011, also Balog along with other colleges, performed a series of MD simulations to deepen the analysis of the previously studied systems. Even if the interesting results collected were a product of the analysis of simulations that represented much more hydrates solutions from those studied with neutrons.
%Questo tipo di complicazione fa si che, anche se si svolgono esperimenti reali su campioni in polvere, molto spesso si simulano comunque sistemi idratati. Ad esempio, nel 2004, Balog et al. fecero un esperimento di neutron scattering e mostrarono che una patricolare proteina nella forma holo vibra più intensamente alle basse frequenze (fra $500$ e $50$ GHz) rispetto alla stessa proteina nella forma apo. Nel 2011, sempre Balog insieme ad altri colleghi, realizzò una serie di MD simulazioni per approfondire l'analisi dei sistemi studiato precedentemente. Sebbene i risultati interessanti che ottenerro, di fatto nelle loro simulazioni vennero fatti evolvere dinamicamente dei sistemi che di fatto rappresentavano soluzioni molto più idratate di quelle studiate con i neutroni.\footnote{Per ovviare a questo, dopo l'evoluzione dinamica dei sistemi abbondantemente idratati, rimossero l'acqua in eccesso ed effettuarono una minimizzazione localizzata dell'energia utilizzando the ABNR (adopted basis Newton Raphson) method -- a minimization method applied in a small subspace of the molecule.}

The methodology used by Balog et al., initially developed by Loncharichi R. J. and Brooks B. R. \cite{loncharich1990temperature}, allows in fact to obtain barely accurate information to study the internal dynamics of the proteins, however is not adapted to study the properties of the hydrating water that surrounds the protein, of barely hydrated solutions \cite{tarek2000dynamics}.
%La metodologia utilizzata da Balog e colleghi, sviluppata inizialmente da Loncharichi R. J. and Brooks B. R. \cite{loncharich1990temperature}, permette infatti di ottenere informazioni abbastanza accurate per studiare la dinamica interna delle proteine, ma si adatta male per studiare le proprietà dell'acqua di idratazione che circonda le proteine in queste soluzioni poco idratate \cite{tarek2000dynamics}. 

On the other hand, it is possible to solve the problems raised from the low hydration, by simulating eight instead of only one protein \cite{tarek2000dynamics}. In this way, even if the computational cost is much more elevated, it is possible to reduce the volume of the box at its most, without risking that a protein interacts with itself. Even if also in this case, the possibility that empty regions are created is still very high, due to the irregular configuration of the protein. In fact, it is very difficult to find a configuration that is compact enough to prevent the formation of hole and, at the same time, where there is no overlapping among the atoms of the proteins. One of the greatest difficulties consists in positioning three-dimensionally, all eight proteins to obtain a well enough initial state.
%D'altra parte, è possibile risolvere i problemi dovuti alla bassa idratazione simulando otto proteine invece che una sola \cite{tarek2000dynamics}. In questo modo, infatti, anche se ovviamente il costo computazionale risulta più elevato, è possibile ridurre il volume della box il più possibile senza rischiare che una proteina interagisca con se stessa. Tuttavia, la possibilità che si creino dei vuoti fra le proteine è ancora molto elevata. A causa della forma irregolare delle proteine, infatti, diventa molto difficile trovare una configurazione compatta abbastanza da impedire la formazione di hole e in cui, allo stesso tempo, ci siano delle sovrapposizioni fra gli atomi delle proteine. Una delle maggiori difficoltà consiste proprio nel disporre tridimensionalmente le otto proteine per ottenere un buono stato iniziale. \\

!!!!!!!!!!!!
Per questo motivo, simulazioni di proteine in polvere in letturatura sono molto poche. Un punto di riferimento importante è l'articolo di Tarek M. e Tobias D. J.: ``\textit{The Dynamics of Protein Hydration Water: A Quantitative Comparison of Molecular Dynamics Simulations and Neutron-scattering Experiments}'' where the authors present an extensive molecular dynamics simulation study of protein hydration water, at a series of temperatures in cluster, crystal, and powder environments \cite{tarek2000dynamics}.
Seguendo quanto fatto da Tarek e Tobias quello che in genere si fa per simulare sistemi in polvere è di replicare otto volte la struttura di una proteina e disporre nello spazio le varie copie della proteina secondo il reticolo cristallino del campione che è stato utilizzato per determinare la struttura della proteina di partenza (solitamente questa informazione è contenuta nel file PDB della proteina ricavabile dal sito del Protein Data Bank -- vedi sezione \ref{sssec:PDB}) \cite{rahaman2017configurational}. Dopo di che si aggiunge l'acqua intorno alle otto proteine, cercando i non lasciare spazi vuoti.

\subsection{The initial state of the simulations}



In questi casi, comunque l'articolo che viene preso come riferimento è solitamente quello 

 They clearly show that simulations of protein hydration performed on ``cluster'' systems, in which a single protein molecule is surrounded by a finite hydration shell, lead to different water behavior compared with powder environments. 


Questi valori sono stati scelti per permettere, in futuro, di utilizzare le simulazioni realizzate nel corso del presente lavoro per analizzare in dettaglio i risultati di alcune misure fatte con i neutroni by Paciaroni and colleges, not yet published.
Su questi sistemi sono state fatte di recente delle misurazioni di scattering di neutroni performed by Paciaroni and colleges, not yet published. Per questo, i sistemi simulati, non solo forniranno fin da subito nuove informazioni, ma potranno anche essere utilizzati in futuro per analizzare in dettaglio i risultati delle misure fatte con i neutroni.
 Inoltre, su questi sistemi sono state fatte di recente delle misurazioni di scattering di neutroni performed by Paciaroni and colleges, not yet published. Perciò, oltre ai dati originali su questo tipo di sistemi ottenuti in questa tesi, le simulazioni qui sviluppate potranno essere utilizzate in futuro anche per analizzare in dettaglio i risultati delle misure fatte con i neutroni. 
Il motivo per cui si è deciso di simulare questo tipo di sistemi è che 
I valori di idratazione scelti per la simulazione dei due sistemi sono stati fissati in modo tale da essere consistenti con quelli di campioni fisici su cui sono state svolte di recente delle misurazioni di scattering di neutroni, not yet published, performed by A. Paciaroni and colleges at ILL. In questo modo, 

 In più, 

and 
in modo tale da fornire dei che permetterann al confronto con misure sperimentali, i quali sono  delle misure di scattering di neutroni con le quali sarà possibile confrontare i dati
MBP and Maltose have been chosen in order to compare some of the simulations' results with the measurements of inelastic neutron scattering performed on these systems by A. Paciaroni and colleges at ILL, not yet published.


La simulazione di sistemi poco idratati come questi Si è deciso di realizzare questo tipo di simulazioni L'utilità di questo tipo di simulazioni Questo tipo di simulazioni La Maltose Binding Protein (nota anche come Maltosedextrin Binding Protein -- abbreviata MBP). Nella sua holo form, l'MPB è legata con una moleca di maltosio (MBP+Malt) in un'ambiente in polvere e


Lo scopo di questa tesi è infatti proprio quello di realizzare delle simulazioni di una particolare proteina, la Maltose Binding Protein (nota anche come Maltosedextrin Binding Protein -- abbreviata MBP), in un'ambiente in polvere con un'idratazione del 38\% (rapporto in massa fra il solvente ed il soluto, rispettivamente, acqua ed MBP) e d

%In this sense, computer simulations that are built upon the actual experimental data, can overcome the limitations of neutron scattering and grasp the more complex spectra of the proteins.\footnote{Some interesting applications, relevant to the interpretation of quasielastic neutron scattering and concerning the simulation-based development of models for slow protein dynamics, are discussed in Kneller's lecture \cite{ref:QNS_Keller}.}
