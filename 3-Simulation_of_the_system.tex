\chapter{Simulation of the system}
Nel presente capitolo sono illustrate le metodologie ed i risultati del lavoro sperimentale di questa tesi. 

In particolare, i sistemi prevalentemente studiati sono soluzioni di proteine in polvere, ovvero poco idratati. Oltre che da un punto di vista puramente scientifico, lo studio di questo tipo di sistemi risulta molto ineteressante per quanto riguarda, per esempio, campi di ricerca applicata che concernono lo studio e la progettazione di farmaci in cui il principio attivo è costituito da proteine\footnote{With the significant advancements in biologics and biopharmaceutical over the years, peptides and proteins have emerged with a host of new applications in the diagnostic as well as the therapeutic sector. As per the current calculations, the market for peptide and protein drugs is estimated around 10\% of the entire pharmaceutical market and will make up an even larger proportion of the market in the future. Since early 1980s, a total 239 therapeutic proteins and peptides are approved for clinical use by US-FDA (United States -- Food and Drug Administration \cite{\url{10.1371/journal.pone.0181748}}}. D'altra parte, questo tipo di sistemi in polvere, sono solitamente molto studiati con la spettroscopia neutronica che è in grado di fornire molte informazioni, in particolare sulla dinamica delle proteine.  

%Thermal neutron scattering is sensitive to the dynamics and the structure of condensed matter on the atomic scalegives precise information about atomic fluctuations and dynamics in proteins, enhancing an average view of all atomic contributions.
%However, due to the high complexity of these systems, usually their internal dynamics is too complicated for a quantitative interpretation. 

%Computer simulations, and in particular Molecular Dynamics (MD) simulations, enable to gain a wider insight into the dynamics of proteins that are not provided from neutron scattering and in general from any real experiments. The combination of both methods access in the same time and space domains, and the comparison of simulated and measured spectra is thoroughly direct, since neutrons are diffused by the atomic nuclei (neglecting magnetic scattering), which are the actual objects of study of MD simulations. Once an alignment between simulated and experimental spectra is found, thus the experimental data supports the data obtained from the computer simulation, then the simulated trajectories can be analyzed in detail and information not accessible from experiments can be extracted from computer simulations.

%In this sense, computer simulations that are built upon the actual experimental data, can overcome the limitations of neutron scattering and grasp the more complex spectra of the proteins.\footnote{Some interesting applications, relevant to the interpretation of quasielastic neutron scattering and concerning the simulation-based development of models for slow protein dynamics, are discussed in Kneller's lecture \cite{ref:QNS_Keller}.}
