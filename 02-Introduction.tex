\chapter*{Introduction} 
\addcontentsline{toc}{chapter}{Introduction} % aggiunge la bibliografia all'indice

Environmental economics is at the forefront of the contemporary issues of society. Human well-being and its surroundings are the target of environmental economics. It is a field that needs to take into account a multitude of conditions that have been studied by the new generations of economists. Most of these studies are performed for well established and controlled markets by the State. This category of markets share a common characteristic: their legality. There are however human activities that were forbidden because they threaten the security or health of the population, but prohibition has not deter their evolvement  and has created markets that conceal their activities from State control. This assumption is also true for the environmental economic field. Illegal markets escape their inclusion from certain branches of the economic theory and as a consequence uninformed public policies generate a myriad of negative externalities; like violence, contamination, unnecessary social costs and heavy government expenditures.

Illegal markets and environmental economics are two fields that combined are understudied.
Lawless markets have different characteristics and conditions, this solely field is too wide in combination with environmental economics for a thesis study.
There are at least four different types of illegal markets: 
\begin{itemize}\justifying
\item[$\circ$] Market exchange of products whose production or provision is as such illegal.
\item[$\circ$] Products or services which are as such legal but whose exchange on markets is outlawed.
\item[$\circ$] Market exchange of products can be illegal because the product offered has been stolen or has been forged.
\item[$\circ$] Markets that violate legal stipulations in the production process.
\end{itemize} 
To limit the extent of this study we will focus on one case among the four types of illegal markets, i.e. drug-trafficking. \\
We will analyze the drugs market and its impact in the contemporary studies of environmental economics and in the conclusions we will summarize all the findings to nominate a deliberate system of principles to guide decisions and achieve a rational outcome from an environmental economic optic. This topic has not been reviewed in the literature as such and as a consequence a new methodology had to be adopted to reach the scopes of this research thesis. The methodology as such is fully described in section \ref{sec:Methods}, it is broadly divided in four stages, for each drug category (cannabis, opium/opiates, Amphetamine Type Stimulants and coca):

\begin{enumerate}
\item Plantation, where deforestation and degradation of the soil will be calculated.
\item Manufacturing, where the waste mass produced will be assessed.
\item Wholesale distribution and the calculation of the emissions produced.
\item Retail distribution and the calculations of the emissions produced.
\end{enumerate}

Economic activities are essential for the subsistence of social communities, yet not all of these activities support longstanding operations with a positive social effect. This would be the case of the exchange of psychoactive substances and narcotics. Narcotraffics is an illegal market with global reach. Drug production and consumption requires a global net that involves many actors, technological development, information flow and efficient transportation means. Globalization has enabled the internationalization of criminal organizations by means of the interconnection of people with the latest communication technology. Getting global means being able to reach out a wider consumer market that increases the organizations income and allows the diversification of activities using the know-how of the business. That is how criminal organization's activities incur in a wider spectra of illegal activities that produce an even larger number of negative externalities and direct harms to society. Such is the case of drug's cartels in Mexico, that according to Buscaglia, have expanded their economic sphere into some 22 different types of crimes besides drug-trafficking. These organizations started trafficking drugs, then producing them, and then using that income to invest in the diversification of profitable crimes, like prostitution, person trafficking, extortion or kidnapping. In the Middle East, the Talibans use heroin money to finance religious and petroleum wars with Western countries. Narcotraffics is much more than just drug trafficking, their reach has become immense and technology has helped their expansion. 

The present research is relevant to understand the consequences of drug-trafficking in the environment for public policy assessment, this topic in particular has not been addressed from this perspective, especially not tackled with such a set up. The results will show that each type of drug works differently and comprises a set of characteristics of each drug market that makes them very diverse from each other, at analytical level this impacts the results.

The objectives of this thesis is first to understand the dimension of the environmental impact of narcotraffics. Once attained this goal it will be possible to pose an opinion on the most suited public policy to address the environmental harms produced from drug trafficking through market regulations.

The hypothesis states that narcotraffics is an outlawed activity with a high impact in the environment that should be regulated by the state through policies that eliminate the negative externalities derived from their illegal condition.

The thesis is divided in three chapters with several sections to enhance a better logical follow-up of the information. \\
The first chapter is entitled \textit{The upsurge of the environmental economic debate}, which is divided in three further subsections: global warming and its impact in the economy, basic concept of environmental economics that outlines the notion of the balance of the materials, efficiency and the cost and benefit analysis which will be retaken in the conclusions. The third subsection depicts the policy instruments commonly used in environmental economics, like pollution charges, tradable permits in particular the cap and trade system, the market friction reduction and finally the incentives.

The second chapter with the title \textit{The economics of drug trafficking} is an outline on a new branch of study in economics, recently labeled by university articles and scholars as \textit{narconomics}, this is a new field of study dedicated to the market of drugs. This chapter is divided into the four categories of drugs, i.e. cannabis, opium/opiates, Amphetamine Type Stimulants (ATS)  and coca, that each one of them constitute separate units of logical analysis comparable to the goods market of different product types.\\
The third and last chapter assembles narcotraffics and the environmental economic analysis, it is divided in a description of the tables of contamination proxy by drug category, followed by a detailed description of the methodology, the last section displays and describes the results assembled from the calculations obtained following that framework.
The conclusions are divided into the results obtained from the \textit{4 stages method} compiled in tables that proxy the contamination effects of the drugs markets by drug category. The second part of the conclusion deals with a proper respond to our initial hypothesis, where the fundamentals of a  suitable policy recommendation are exposed to deal with the issue of drug trafficking from an environmental economic perspective based on the results attained from this research thesis.
 

%I dwell upon the day when the ex president of Mexico declared the war on drugs in 2006, it was just ten days after assuming the presidency of the country. All national programs were interrupted to broadcast Felipe Calder\'{o}n, the ten-day old president, announcing civil war without second thoughts. His exact words were: ``I know this is going to cost a lot of money and many deaths, but it's a necessary thing to do''. Millions of Mexicans watched his declaration of war without actually imagining the implications of such an announcement, but understanding came with the material consequences of that declaration of war: 178 262 dead people linked to drugs in the period 2006-2015 (without actually taking into account the disappearances that amount around 32 000 people, estimations are variable, and it is very possible that the number is much higher due to the lack of trust in the Mexican judicial system, as Mexicans avoid police reports). Mexican society looks back at Felipe Calder\'{o}n's presidency and regrets the deaths and cries over the lost life's, many are still digging in the common graves trying to find the bones of their disappeared family members. Numerous Mexicans associate the very name of Calder\'{o}n as a synonym of chaos and dead.\textit{Narcotraffics} has permeated Mexican society to the bones, it is present in the \textit{narcocorridos}, music made for narcos; \textit{narconovelas}, a new boom in the Latin American TV shows dedicated to the life of narcotraffickers like \textit{El Chapo} or Pablo Escobar, which have reached record TV ratings in later years; the worshipping of Malverde, the saint of the \textit{narcotraffickers}, who commend themselves to a church dedicated to him in Tijuana before passing drugs to ``the other side'', when they pray not to get caught. Children's games have also changed: plastic guns are commonly sold in markets and play of being \textit{sicarios}, hunting and killing each other, the one who kills the most, wins. \textit{Narcos} have changed Mexican society, like the Revolution of 1910, a collective trauma is being created that will become a shadow in the generations to come, many question themselves: the revolution lasted for 10 years, with a visible ending, and 11 years have already pass since war was declared on the \textit{narcos de facto} but not \textit{di jure}\footnote{The Mexican constitution foresees the application of Article 29 which suspends all human rights and enters in state of emergency in case of war, but this article has not been used since it was created in 1917, instead an unconstitutional law entered into force in 2017 that gives the direct power to the executive of the military corpse, with no counterbalance of the Chambers, the Mexican society protested, but their claims were not taken into account. This law is similar to the emergency law Hitler used to take absolute power of the German State.}, but this time, war seems endless and deeply rooted in the Mexican culture, so when will it end?
%Is there a visible deal that can bring peace to the ambition of the \textit{narcos}?
%Is there going to be relieve for those who still expect their love ones to enter after years of being disappeared?

%Not only dead people, but the disappearances have a second effect on society. Hopeless hope, endless search and incomplete graving, this a new figure in mass that Mexicans have just begun to realize.
%The further I dig into the world of drug trafficking the more I realize how enormous this problem has become, not only in Mexico, but in many societies, all countries are affected from the demand or supply side of drugs in a higher or lesser degree. At a worldwide level 1\% of the global GDP is generated by drugs, this is a lot of money and a lot of families whose main income comes from drug trafficking. But this issue is not only affecting people's lives through the increase in the violence levels, corruption and insecurity, but also in ways usually taken as, to say the least, less important. One of the issues that has been at the stage of legal economies is the environmental concern, notwithstanding the efforts of many countries to find a solution, the calculations and research are mainly directed towards those entities that are visible to governments. But what if the invisible becomes visible through a change in the actual prohibition pattern? What if the right policy can solve a part of this problem? \textit{Narcotraffics} may be a local problem, like the violence in Mexico, the wars financed in the Middle East though opium-money, but the environment is not confined to a place, it is affecting all, poor and rich alike, plants and humans as well.

%People and newspapers confirm that the weather has changed, that in later years weather has become more extreme and that predicting it has become harder for farmers and people everywhere.

%In general, Mexicans are quite unaware of the damage people yield in the environment, it has to do with the fact that they are firstly concerned about covering their basic needs like food and housing rather than taking care of the environment. Usually environmental consciousness is more usual in developed countries because people have the resources to think at the following stage of livelihood, which implies taking informed action. The basic ingredients for a broader environmental awareness are: money, education and time, which are quite limited resources in developing countries.

%Being aware that humans are only a part of the system and that nature is our mean for survival, always considering the permanent interaction between humans and environment, one affecting the other underlies any further analysis. Nonetheless the greatest economic powers are careless of the environment and careful of their profits, they seem to disregard the fact that the cleaner the environment the higher the quality of life that is common to all, Europeans, Americans or Africans.

%There is no other way to correct desires and ``market distortions'' than through incentives, limits or prohibitions, and that is how the economy tries to adjust the market and minimize the negative externalities, sometimes it ends up correcting a few and creating new ones, thereby the story of economics becomes endless. But so far, this is the way society has found to limit itself from self-destruction. Humanity has got itself on an edge, where choices are becoming increasingly limited as we try to look away from the environmental problem. The harder we push on economic growth the more we provoke environmental damage, the higher the GDP\footnote{Gross domestic product.} growth the more we destroy nature. In this sense growth does not equal development or quality of life, economic growth is a solely searched objective. GDP does not follow the logic of human development; it is just a number that indicates that countries produce an amount, that later on is consumed in an uniformed way.

%In the groove of the environmental problem we find the non-regulated economies, meaning illegal markets. In order to escape from the government's eyes, many of these hidden economic activities use highly polluting systems that have undoubtedly an impact in the environment. Incredibly many of the solutions governments use to contest illegal activities are also highly contaminating, for example plant eradication or war.

%In the mind of a drug-lord the only thing that matters is money, no matter the consequences, thus it would never esteem the environmental impact of a business decision, but ignoring it doesn't mean there is none. In fact, as we will show throughout the thesis argumentation, their impact is considerably higher than expected. Drug traffickers use the most polluting means of transport and sowing techniques: submarines, helicopters and air fumigation over the crops.

%Drug trafficking affects the environment in many ways, one of them is through agriculture, hence the most consumed drugs are plant products like Marijuana, a market that reaches out to 4.2 million people around the world, in spite of its illegal condition. 

%According to \textit{Proceso}, one of the most prestigious magazines in the Mexican media, 30\% of all cultivable farmland in Mexico is dedicated to cultivate poppy and marijuana plants and evidence has shown that \textit{narco} farmers are more attached to traditional farming than to newer techniques, this has many negative environmental implications that will be further detailed in the following chapters.% (Inserire referenza: \url{http://www.proceso.com.mx/93189/la-tierra-de-los-narcos}).

%Insofar, these are my reasons for writing about Narcotraffics and environmental economics; about the way these two topics are intertwined I dedicated the third chapter of this thesis to explain that precise correlation.

%The theme of this thesis refers to the material evidence that shows that the contamination derived from drug trafficking should be considered as an axiom and taken into account in the calculations of models in environmental economics.

%The present study will be dedicated to proof if the contamination resultant from the drug production and commercialization should be taken as a meaningful value in models of environmental economics.

%The study will not be confined to a specific region, but Mexico will be taken as one of the principal references in drug production and commercialization, do to its raising importance in this illegal market and the availability of literature that was used for this research. Regarding the temporality of this study, actually no specific frame of time is taken as a starting point. Though the later years, namely the last eleven years (from 2006 until 2017) will be found more frequently in order to find the later available trends in both environmental and narco-economics.

%The first chapter is an approximation to Environmental Economics, thus the roots of this increasingly important area within economics, that takes upsurge from the idea of climate change as an axiom and that humans do impact the environment that has to be regulated and an analysis of the optimal level of constrain is at the stage of world agreements, like the Paris Agreement born in the rooms of the United Nations a few years ago. The approach to Environmental Economics outlines ideas like the Pareto Efficiency, the Kaldor-Hicks Efficiency, the balance of the materials, the economical pollution and the types of market and non-market based instruments to regulate the emissions produced by the economies.

